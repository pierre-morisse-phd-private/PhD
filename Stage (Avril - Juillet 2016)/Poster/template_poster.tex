
\documentclass[portrait,a0b,final]{a0poster}




\usepackage{times}
\usepackage{epsfig}
\usepackage{graphicx}
\usepackage[usenames]{color}
\usepackage{multicol}
\usepackage{pstricks,pst-grad}
\usepackage{amsmath,amssymb}
\usepackage{amscd}
\usepackage{color}
\thispagestyle{empty}

\newgray{grispale}{0.7}
\newgray{grismoy}{0.5}
%\newrgbcolor{LemonChiffon}{1. 0.98 0.8}
\newrgbcolor{LightBlue}{0.68 0.85 0.9}
\newrgbcolor{lblue}{0.8 0.92 0.95}
\newrgbcolor{lred}{1 0.8 0.8}
\definecolor{lgreen}{rgb}{0.2,0.7,0.2}
\newrgbcolor{lyellow}{1 1 0.6}
\newrgbcolor{orange}{1 0.7 0.2}
\newrgbcolor{lorange}{0.9 0.9 0.7}
\newrgbcolor{lgreen}{0.87 0.95 0.8}
\newrgbcolor{dgreengray}{0.37 0.43 0.37}
\newrgbcolor{dgreen}{0.36 0.75 0.41}
\newrgbcolor{violet}{0.7 0.4 0.7}

\newcommand{\RR}{{\ensuremath{\mathbb R}}}
\newcommand{\ZZ}{{\ensuremath{\mathbb Z}}}
\newcommand{\LL}{{\ensuremath{\mathrm L}}}
\newcommand{\HH}{{\ensuremath{\mathrm H}}}

%%%%%%%%%%%%%%%%%%%%%%%%%%%%%%%%%%%%%%%%%%%
% Definition of some variables and colors
\setlength{\columnsep}{3cm}
\setlength{\columnseprule}{2mm}
\setlength{\parindent}{0.0cm}

%%%%%%%%%%%%%%%%%%%%%%%%%%%%%%%%%%%%%%%%%%%%%%%%%%%%
%%%               Background                     %%%
%%%%%%%%%%%%%%%%%%%%%%%%%%%%%%%%%%%%%%%%%%%%%%%%%%%%
\newcommand{\background}[3]{
  \newrgbcolor{cgradbegin}{#1}
  \newrgbcolor{cgradend}{#2}
  \psframe[fillstyle=gradient,gradend=cgradend,
  gradbegin=cgradbegin,gradmidpoint=#3](0.,0.)(1.\textwidth,-1.\textheight)
}

%%%%%%%%%%%%%%%%%%%%%%%%%%%%%%%%%%%%%%%%%%%%%%%%%%%%
%%%                Poster                        %%%
%%%%%%%%%%%%%%%%%%%%%%%%%%%%%%%%%%%%%%%%%%%%%%%%%%%%
\newenvironment{poster}{
  \begin{center}
  \begin{minipage}[c]{0.80\textwidth}
}{
  \end{minipage}
  \end{center}
}

%%%%%%%%%%%%%%%%%%%%%%%%%%%%%%%%%%%%%%%%%%%%%%%%%%%%
%%%                pbox                          %%%
%%%%%%%%%%%%%%%%%%%%%%%%%%%%%%%%%%%%%%%%%%%%%%%%%%%%
\newrgbcolor{lcolor}{0. 0. 0.80}
\newrgbcolor{gcolor1}{1. 1. 1.}
\newrgbcolor{gcolor2}{.80 .80 1.}

\newcommand{\pbox}[4]{
\psshadowbox[#3]{
\begin{minipage}[t][#2][t]{#1}
#4
\end{minipage}
}}

%%%%%%%%%%%%%%%%%%%%%%%%%%%%%%%%%%%%%%%%%%%%%%%%%%%%
%%%                myfig                         %%%
%%%%%%%%%%%%%%%%%%%%%%%%%%%%%%%%%%%%%%%%%%%%%%%%%%%%
% \myfig - replacement for \figure
% necessary, since in multicol-environment
% \figure won't work
\newcommand{\myfig}[3][0]{
\begin{center}
  \vspace{1.5cm}
  \includegraphics[width=#3\hsize,angle=#1]{#2}
  \nobreak\medskip
\end{center}}

%%%%%%%%%%%%%%%%%%%%%%%%%%%%%%%%%%%%%%%%%%%%%%%%%%%%
%%%                mycaption                     %%%
%%%%%%%%%%%%%%%%%%%%%%%%%%%%%%%%%%%%%%%%%%%%%%%%%%%%
% \mycaption - replacement for \caption
% necessary, since in multicol-environment \figure and
% therefore \caption won't work

%\newcounter{figure}
\setcounter{figure}{1}
\newcommand{\mycaption}[1]{
  \vspace{0.5cm}
  \begin{quote}
    {{\sc Figure} \arabic{figure}: #1}
  \end{quote}
  \vspace{1cm}
  \stepcounter{figure}
}

%qq cadres pour les titres
%on applique les couleurs que l'on veut
\newcommand{\titreic}[1]{\begin{center}\pbox{0.5\columnwidth}{}{linewidth=2mm,framearc=0.1,linecolor=lightblue,fillstyle=gradient,gradangle=0,gradbegin=white,gradend=whiteblue,gradmidpoint=1.0,framesep=1em}{\begin{center}\LARGE #1\end{center}}\end{center}\vspace{1.5cm}}

\newcommand{\titrebleu}[1]{\begin{center}\pbox{0.6\columnwidth}{}{linewidth=2mm,framearc=0.1,linecolor=blue,fillstyle=gradient,gradangle=0,gradbegin=white,gradend=lblue,gradmidpoint=1.0,framesep=1em}{\begin{center}\LARGE #1\end{center}}\end{center}\vspace{1.5cm}}

\newcommand{\titrerouge}[1]{\begin{center}\pbox{0.6\columnwidth}{}{linewidth=2mm,framearc=0.1,linecolor=red,fillstyle=gradient,gradangle=0,gradbegin=white,gradend=red,gradmidpoint=1.0,framesep=1em}{\begin{center}\large #1\end{center}}\end{center}\vspace{1.5cm}}

\newcommand{\titrevert}[1]{\begin{center}\pbox{0.6\columnwidth}{}{linewidth=2mm,framearc=0.1,linecolor=green,fillstyle=gradient,gradangle=0,gradbegin=white,gradend=green,gradmidpoint=1.0,framesep=1em}{\begin{center}\large #1\end{center}}\end{center}\vspace{1.5cm}}

\newcommand{\titreorange}[1]{\begin{center}\pbox{0.6\columnwidth}{}{linewidth=2mm,framearc=0.1,linecolor=orange,fillstyle=gradient,gradangle=0,gradbegin=white,gradend=orange,gradmidpoint=1.0,framesep=1em}{\begin{center}\LARGE #1\end{center}}\end{center}\vspace{1.5cm}}

\newcommand{\sstitre}[1]{\begin{center}\bf \Large #1 \end{center}\vspace{1cm}}


\newtheorem{defn}{Definition}
\newtheorem{prop}{Proposition}
\def\ia{\rightarrow\hspace{-40pt}+\ }
\def\ca{\nearrow\hspace{-32pt}\searrow}
\def\cs{\rightarrow\hspace{-15pt}+\ }
\def\SGS{\bigoplus_{n\geq0}{\mathbb Q}[\SG_n]}
\def\FQSym{{\bf FQSym}}
\def\MQSym{{\bf MQSym}}
\def\Zy{\frak Z}
\def\F{\mathbf{F}}
\def\ncp#1#2{#1\langle #2 \rangle}
\def\ncs#1#2{#1\langle\langle #2 \rangle\rangle}
\def\G{\frak{G}}
\def\SG{\frak{S}}
\def\si{\sigma}
\def\om{\omega}
\def\al{\alpha}

\def\N{\mathbb{N}}
\def\Q{\mathbb{Q}}
\def\Z{\mathbb{Z}}
\def\p{\frak p}


%%%%%%%%%%%%%%%%%%%%%%%%%%%%%%%%%%%%%%%%%%%%%%%%%%%%%%%%%%%%%%%%%%%%%%
%%% Begin of Document
%%%%%%%%%%%%%%%%%%%%%%%%%%%%%%%%%%%%%%%%%%%%%%%%%%%%%%%%%%%%%%%%%%%%%%
\begin{document}


\newrgbcolor{lightblue}{0. 0. 0.80}
\newrgbcolor{white}{1. 1. 1.}
\newrgbcolor{whiteblue}{.80 .80 1.}


\begin{poster}

%%%%%%%%%%%%%%%%%%%%%
%%% Header
%%%%%%%%%%%%%%%%%%%%%
\begin{center}
\titreorange{\textbf{\huge Free Quasi-Symmetric Functions, Product
Actions and Quantum Field Theory of Partitions\\}}

\vspace*{1cm}

\textsl{\LARGE G. H.E. Duchamp$^1$ J.-G. Luque$^2$, K. A. Penson$^3$ and  C. Tollu$^1$}

\vspace*{1cm}

\begin{tabular}{ccc}

\begin{minipage}[b]{0.3\textwidth} \large %\center

      $^1$ Institut Galil\'ee, LIPN\\
      Universit\'e Paris 13\\
      F- 93430 Villetaneuse, France\\
      e-mail: \textsf{\{ghed,ct\}@lipn-univ.paris13.fr}
    \end{minipage} &


    \begin{minipage}[b]{0.3\textwidth} \large %\center

      $^2$ Institut G.~Monge  UMR-CNRS 8049\\
      Univ. Marne la Vall{\'e}e\\
      F-77454 Marne la Vall{\'e}e, Cedex 2, France \\
      e-mail: \textsf{luque@univ-mlv.fr}
    \end{minipage} &



    \begin{minipage}[b]{0.3\textwidth} \large %\center

      $^3$ LPTCM CNRS UMR 7600\\
      Univ. Marne Pierre et Marie Curie,\\
       F 75252 Paris Cedex 05, France. \\
      e-mail: \textsf{penson@lptl.jussieu.fr}
    \end{minipage}
  \end{tabular}
\end{center}

\vspace*{3cm}

%%%%%%%%%%%%%%%%%%%%%
%%% Content
%%%%%%%%%%%%%%%%%%%%%
%%% Begin of Multicols-Enviroment
%ici 2 colonnes par page
\begin{multicols}{2}

\Large \titreic{Introduction} Dans un papier relativement r�cent, Philippe et al. ont soulign� l'importance d'indexer les reads afin de r�soudre des probl�mes de mapping ou de correction, et ont d�velopp�e un index supportant les requ�tes suivantes : \\

\begin{itemize}
	\item Dans quels reads $f$ appara�t ?
	\item Dans combien de reads $f$ appara�t ?
	\item Quelles sont les occurrences de $f$ ?
	\item Quel est le nombre d'occurrences de $f$ ?
	\item Dans quels reads $f$ n'appara�t qu'une fois ?
	\item Dans combien de reads $f$ n'appara�t qu'une fois ?
	\item Quelles sont les occurrences de $f$ dans les reads o� $f$ n'appara�t qu'une fois ? \bigskip
\end{itemize} 

\titrebleu{Actions of a direct product of permutation groups}
\sstitre{Direct product actions}
Two pairs $(G_1,X_1)$ and $(G_2,X_2)$,  each $G_i$ is
a permutation group acting on $X_i$.\\
 {\it \color{red} Intransitive action}
of $G_1\times G_2$ on $X_1\sqcup
 X_2$ :
\begin{equation}
(\sigma_1,\sigma_2)x=\left\{\begin{array}{ll}\sigma_1x&\mbox{ if
}x\in X_1\\ \sigma_2x&\mbox{ if }x\in X_2
\end{array}\right. .\nonumber
\end{equation}
 $(G_1,X_1)\ia(G_2,X_2):=(G_1\times G_2,X_1\sqcup
 X_2)$.\\
  {\it \color{red} Cartesian action} of
$G_1\times G_2$ on $X_1\times X_2$:
\begin{equation} (\sigma_1,\sigma_2)(x_1,x_2)=(\sigma_1
x_1,\sigma_2 x_2).\nonumber
\end{equation}
$(G_1,X_1)\ca(G_2,X_2):=(G_1\times G_2,X_1\times X_2)$.\\ \\
\sstitre{Explicit realization}
 Denote
 \begin{enumerate}
 \item[$\bullet\ \ $] by $\circ_N$ the natural action of $\SG_n$ on $\{0,\dots,n-1\}$,
\item[$\bullet\ \ $]  by $\circ_I$ the intransitive action of $\SG_n\times \SG_m$ on
$\{0,\cdots,n+m-1\}$
\item[$\bullet\ \ $] by $\circ_C$ the cartesian action of $\SG_n\times \SG_m$ on $\{0,\dots,nm-1\}$.
\end{enumerate}
  More precisely,
\begin{equation}
(\sigma_1,\sigma_2)\circ_Ii=\left\{\begin{array}{ll}
\sigma_1\circ_Ni&\mbox{if }0\leq i\leq n-1\\
\sigma_2\circ_N(i-n)+n&\mbox{if }n\leq i\leq
n+m-1\end{array}\right. \nonumber.
\end{equation}
and
\begin{equation}
(\sigma_1,\sigma_2)\circ_C(j+nk)=(\sigma_1\circ_Nj)+n(\sigma_2\circ_Nk)\nonumber
\end{equation}
for $0\leq i\leq n+m-1$, $0\leq j\leq n-1$ and $0\leq k\leq m-1$.\\
Let the map
$\ia:\SG_n\times\SG_m\rightarrow \SG_{n+m}$ defined by
\begin{equation}
\sigma_1\ia\sigma_2=\sigma_1\sigma_2[n]\nonumber
\end{equation}
\pbox{0.9\columnwidth}{}{linewidth=2mm,framearc=0.1,linecolor=lblue,fillstyle=gradient,gradangle=0,gradbegin=white,gradend=white,gradmidpoint=1.0,framesep=1em}{
 $\sigma_1={\color{blue}1320}\in\SG_4$,  $\sigma_2={\color{red}534120}\in\SG_6$.
\begin{equation}
\sigma_1\ia\sigma_2={\color{blue}1320}{\color{red}978564}\nonumber,\,
\sigma_2\ia\sigma_1={\color{red}534120}{\color{blue}7986}\nonumber
\end{equation}
}
\newpage
{\bf Proposition}\\
$(\sigma_1\ia\sigma_2)\circ_Ni=(\sigma_1,\sigma_2)\circ_Ii .$\\ \\
Let the map $\ca:\SG_n\times\SG_m\rightarrow \SG_{nm}$ defined by
\begin{equation}\label{defcar}
\sigma_1\ca\sigma_2=\prod_{i,j}c_i\ca c'_j\nonumber
\end{equation}
where $\sigma_1=c_1\cdots c_{k}$ and $\sigma_2=c'_1\cdots c'_{k'}$
are
 the decompositions of $\sigma_1$ and $\sigma_2$ in a product
of cycles
and \begin{equation}\label{cycleintr} c\ca
c'=\prod_{s=0}^{l\wedge l'-1}(\phi(s,0),\phi(s+1,1)\cdots,
\phi(s+l\vee l'-1,l\vee l'-1)),\nonumber
\end{equation}
($\wedge:=$  gcd,  $\vee:=$  lcm, $c=(i_0,\cdots, i_{l-1})$,
$c'=(j_0,\cdots,j_{l'-1})$ are two cycles and
$\phi(k,k')=i_{k\mbox{ mod }l}+nj_{k'\mbox{ mod }l'}$.)\\
 The cartesian action is
compatible with the natural action.\\
{\bf Proposition}\\
$(\sigma_1\ca\sigma_2)\circ_Ni=(\sigma_1,\sigma_2)\circ_Ci\ .$\\ \\
\pbox{0.9\columnwidth}{}{linewidth=2mm,framearc=0.1,linecolor=lblue,fillstyle=gradient,gradangle=0,gradbegin=white,gradend=white,gradmidpoint=1.0,framesep=1em}{ $c_1=(0,2,3,1)$, $c_2=(7,6,5,4,3,2)$.\\
 $c_1\ca c_2:$\\
\begin{center}
\setlength{\unitlength}{2mm}
\resizebox{25cm}{17cm}{\begin{picture}(45,30)(-5,-5)
\linethickness{1.5pt}
%axe vertical
\put(0,0){\line(0,1){25}} \put(0,0){\line(1,0){35}}
\put(-1,5){\line(1,0){2}}\put(-1,10){\line(1,0){2}}
\put(-1,15){\line(1,0){2}}\put(-1,20){\line(1,0){2}}
%axe horizontal
\put(5,-1){\line(0,1){2}}\put(10,-1){\line(0,1){2}}
\put(15,-1){\line(0,1){2}}\put(20,-1){\line(0,1){2}}
\put(25,-1){\line(0,1){2}}\put(30,-1){\line(0,1){2}}
%graduation verticale
\put(-3,-1){\footnotesize 0}\put(-3,4){\footnotesize
2}\put(-3,9){\footnotesize 3} \put(-3,14){\footnotesize
1}\put(-3,19){\footnotesize 0}
%graduation horizontale
\put(-1,-3){\footnotesize 7}\put(4,-3){\footnotesize
6}\put(9,-3){\footnotesize 5}\put(14,-3){\footnotesize
4}\put(19,-3){\footnotesize 3}\put(24,-3){\footnotesize 2}
\put(29,-3){\footnotesize 7}
%le cycle 1
\color{blue}
\linethickness{1pt}
\put(0,0){\line(1,1){5}}\put(5,5){\line(1,1){5}}\put(10,10){\line(1,1){5}}
\put(15,15){\line(1,1){5}}\put(20,0){\line(1,1){5}}\put(25,5){\line(1,1){5}}
\put(0,10){\line(1,1){5}}\put(5,15){\line(1,1){5}}\put(10,0){\line(1,1){5}}
\put(15,5){\line(1,1){5}}\put(20,10){\line(1,1){5}}\put(25,15){\line(1,1){5}}
%vertical et horizontal
%\setlinestyle{dotted}
\linethickness{0.25pt} \multiput(20,0)(0,1){20}{\line(0,-1){0.2}}
\multiput(10,0)(0,1){20}{\line(0,-1){0.2}}
\multiput(30,0)(0,1){20}{\line(0,-1){0.2}}
 \multiput(30,10)(-1,0){30}{\line(-1,0){0.2}}
%fl�ches
\put(5,17.5){{\tiny$\nearrow$}} \put(3.8,7){{\tiny$\nearrow$}}
\put(13.8,7){{\tiny$\nearrow$}} \put(23.8,7){{\tiny$\nearrow$}}
% \put(9.5,16){{$\downarrow$}}\put(19.6,16){{$\downarrow$}}
% \put(29.5,16){{$\downarrow$}}
%  \put(16,9.5){{$\leftarrow$}}

 %croix
 \multiput(5,5)(5,0){6}{\circle*{0.5}}
 \multiput(5,10)(5,0){6}{\circle*{0.5}}
 \multiput(5,15)(5,0){6}{\circle*{0.5}}
 \multiput(5,20)(5,0){6}{\circle*{0.5}}
%cycle 2
\color{red}
\linethickness{1pt}
\put(0,5){\line(1,1){15}}\put(15,0){\line(1,1){15}}
\put(0,15){\line(1,1){5}}\put(5,0){\line(1,1){20}}
\put(25,0){\line(1,1){5}}
%lignes
\linethickness{0.25pt} \multiput(5,0)(0,1){20}{\line(0,-1){0.2}}
\multiput(15,0)(0,1){20}{\line(0,-1){0.2}}
\multiput(25,0)(0,1){20}{\line(0,-1){0.2}}
 \multiput(30,5)(-1,0){30}{\line(-1,0){0.2}}
  \multiput(30,15)(-1,0){30}{\line(-1,0){0.2}}
  %fleches
  \put(0.5,17.5){{\tiny$\nearrow$}}\put(10.5,17.5){{\tiny$\nearrow$}}
  \put(20.5,17.5){{\tiny$\nearrow$}} \put(20.5,7.5){{\tiny$\nearrow$}}
  \put(25.5,2.5){{\tiny$\nearrow$}}
  %\put(4.5,17){{$\downarrow$}}\put(14.6,17){{$\downarrow$}}
 % \put(24.5,17){{$\downarrow$}}
  %\put(16,14.5){{$\leftarrow$}} \put(16,4.6){{$\leftarrow$}}

   %croix
 \multiput(5,5)(5,0){6}{\circle*{0.5}}
 \multiput(5,10)(5,0){6}{\circle*{0.5}}
 \multiput(5,15)(5,0){6}{\circle*{0.5}}
 \multiput(5,20)(5,0){6}{\circle*{0.5}}
\end{picture}}\nonumber\\
$\color{blue}(28,26,23,17,12,10,31,25,20,18,15,9)$\\
$\color{red}(30,27,21,16,14,11,29,24,22,19,13,8).$
\end{center}}\\ \\
\sstitre{Algebraic structure} {\bf Proposition}{\it\ \ \
Associativity}\\ Let $\sigma_1\in \SG_n$, $\sigma_2\in\SG_m$ and
$\sigma_3\in\SG_p$ be $3$ permutations
\begin{enumerate}
\item
$\sigma_1\ia(\sigma_2\ia\sigma_3)=(\sigma_1\ia\sigma_2)\ia\sigma_3$
\item $\sigma_1\ca(\sigma_2\ca\sigma_3)=(\sigma_1\ca\sigma_2)\ca\sigma_3$
\end{enumerate}
{\bf Proposition}{\it\ \  Semi-distributivity}\\
 $\sigma_1\in \SG_n$, $\sigma_2\in\SG_m$ and $\sigma_3\in\SG_p$
\[\sigma_1\ca(\sigma_2\ia\sigma_3)=(\sigma_1\ca\sigma_2)\ia(\sigma_1\ca\sigma_3) \]

\newpage

\end{multicols}
\end{poster}

\end{document}
