\documentclass[t,12pt]{beamer}
\usepackage[T1]{fontenc} 
\usepackage[latin1]{inputenc}
\usepackage[british,frenchb]{babel}
\usepackage{pslatex}
\usetheme[secheader]{Boadilla}
\usepackage{amsmath}
\usepackage{graphicx}
\usepackage{tikz}
\usetikzlibrary{arrows}
\usetikzlibrary{patterns}
\usetikzlibrary{decorations.pathreplacing}
\usepackage{multirow}

\def\sl{{\par\leavevmode\par}}

\title{Correction and assembly of long reads}
\date{\today}
\author[Pierre Morisse]{Pierre Morisse}
%
\begin{document}

	\AtBeginSection{
		\begin{frame}
			\tableofcontents[currentsection]
		\end{frame}
	}
	
	\frame{\titlepage}
	
	\begin{frame}[c]{Plan}
		\tableofcontents[hideallsubsections]
	\end{frame}
	
	\section{First correction approach}
	
	\begin{frame}[c]{Idea}
		\begin{itemize}			
			\item Like NaS, produce synthetic long reads from an assembly of short reads \sl
			
			\item Get rid of the alignment of short reads against each other \sl
			
			\item Deduce information only from the mapping of short reads on long reads \sl
		\end{itemize}				
	\end{frame}
	
	\begin{frame}[c]{Workflow}		
		We present the method for the correction of one long read \sl
		
		\onslide<2->{
		\begin{block}{Principle}
				Mapping of the short reads and the template long read, and definition of a
				$lmin$ threshold, to retrieve reads that: \sl
				
				\begin{itemize}
					\onslide<3->{
					\item Are fully aligned, and used as seeds 
					}
					\onslide<4->{
					\item Have an aligned prefix of length at least $lmin$
					}
					\onslide<5->{
					\item Have an aligned suffix of length at least $lmin$
					}
				\end{itemize}
		\end{block}
		}
	\end{frame}
	
	\begin{frame}[c]{Workflow}
		\begin{block}{Two extension steps:}
				\begin{enumerate}
					\onslide<2->{
					\item Recruitment of partially aligned reads, similar to the seeds \sl
					\begin{center}
						\resizebox{0.4\textwidth}{!}{
							\begin{tikzpicture}
								%SEED
								\draw [|-|] (1,1) -- (4,1);
								\node [right] at (4,1) {seed};
								%SUFFS
								\draw [red] [|-|] (-1.5,0) -- (-0.5,0);
								\draw [red] [|-|] (-0.5,0) -- (1.5,0);
								\node [left] at (-1.5,0) {\textcolor{red}{$r_1$}};
								\draw [red] [<->] (-0.5,-0.5) -- (1.5,-0.5);
								\node [below] at (0.5,-0.5) {\textcolor{red}{aligned suffix}};
								%PREF
								\draw [green] [|-|] (2.5,0) -- (4,0);
								\draw [green] [|-|] (4,0) -- (5.5,0);
								\node [right] at (5.5,0) {\textcolor{green}{$r_2$}};
								\draw [green] [<->] (2.5,-0.5) -- (4,-0.5);
								\node [below] at (3.225,-0.5) {\textcolor{green}{aligned prefix}};
								%LMINS
								\draw [<->] (1,0.5) -- (2,0.5);
								\node [above] at (1.5,0.5) {$lmin$};
								\draw [<->] (3,0.5) -- (4,0.5);
								\node [above] at (3.5,0.5) {$lmin$};
							\end{tikzpicture}
						}
					\end{center}
					}
					
					\onslide<3->{
					\item Recruiting of new partially aligned reads, without similarity relation, by fixing
					a new threshold $lmax$ \sl
					\begin{center}
						\resizebox{0.6\textwidth}{!}{
							\begin{tikzpicture}
								%SEED
								\draw [|-|] (0,1) -- (7,1);
								\node [right] at (7,1) {contig};
								
								%LMAX
								\draw [<->] (0,1.5) -- (0.5,1.5);
								\node [above] at (0.25,1.5) {$lmax$};
								\draw [<->] (6.5,1.5) -- (7,1.5);
								\node [above] at (6.75,1.5) {$lmax$};
								
								%SUFF
								\draw [red] [|-|] (-2,0.5) -- (-0.3,0.5);
								\draw [red] [|-|] (-0.3,0.5) -- (1,0.5);
								\node [left] at (-2,0.5) {\textcolor{red}{$s_1$}};
								
								%PREF								
								\draw [green] [|-|] (6.85,0.5) -- (8.25,0.5);
								\draw [green] [|-|] (8.25,0.5) -- (9.85,0.5);
								\node [right] at (9.85,0.5) {\textcolor{green}{$p_1$}};
							\end{tikzpicture}
						}
					\end{center}
					}
				\end{enumerate}
		\end{block}
	\end{frame}
	
	\begin{frame}[c]{Example}
		\resizebox{1\textwidth}{!}{
			\begin{tikzpicture}
				% Etape 1
				% Template
				\fill [pattern=north west lines, pattern color=blue] (-4,1.9) rectangle (-1,2.1);
				\draw [-, blue] (-1,2) -- (0.5,2);
				\fill [pattern=north west lines, pattern color=blue] (0.5,1.9) rectangle (1,2.1);
				\draw [-, blue] (1,2) -- (5,2);
				\fill [pattern=north west lines, pattern color=blue] (5,1.9) rectangle (6.3,2.1);
				\draw [-, blue] (6.3,2) -- (7.3,2);
				\fill [pattern=north west lines, pattern color=blue] (7.3,1.9) rectangle (14,2.1);
				\node [right] at (14,2) {\textcolor{blue}{\footnotesize \emph{template}}};
				
				% Seed
				\onslide<2->{
					\fill[fill=black] (2,0.8) rectangle (5,1);
					\node [right] at (5,1) {\footnotesize seed};
				}
				
				% Read avec suffixe align�
				\onslide<3->{
					\draw [green] [|-] (0,0) -- (1,0);
					\fill [green] (1,-0.1) rectangle (3,0.1);
				}
				
				% Read avec pr�fixe align�
				\onslide<4->{
					\fill [green] (3.5,-0.1) rectangle (5,0.1);
					\draw [green] [-|] (5,0) -- (6.5,0);
				}
				
				% Lmin
				\onslide<3->{
					\draw [<->] (2,0.6) -- (3,0.6);
					\node [below] at (2.5,0.6) {\footnotesize $lmin$};
				}
				\onslide<4->{
					\draw [<->] (4,0.6) -- (5,0.6);
					\node [below] at (4.5,0.6) {\footnotesize $lmin$};
				}
				
				% Fl�che
				\onslide<5->{
					\draw [-triangle 45] (3.25,-1) -- (3.25,-2.5);
				}
				
				% Contig obtenu
				\onslide<5->{
					\draw [green] [|-] (0,-3.5) -- (1,-3.5);
					\fill [green] (1,-3.4) rectangle (2,-3.6);
					\fill[fill=black] (2,-3.4) rectangle (5,-3.6);
					\draw [green] [-|] (5,-3.5) -- (6.5,-3.5);
					\node [right] at (6.5,-3.5) {\footnotesize obtained contig};
				}
				
				% Accolade
				\onslide<6->{
					\draw [decorate,decoration={brace, amplitude=10pt, mirror}]
					(-1,1) -- (-1,-3.6) node [black, midway, xshift=-2cm] 
					{\footnotesize Step 1};
				}
				
				
				% Etape 2
				% Read avec suffixe align�
				\onslide<7->{
					\draw [green] [|-] (-2.5,-4.5) -- (-1,-4.5);
					\fill [green] (-1,-4.4) rectangle (0.5,-4.6);
				}
				
				% Read avec pr�fixe align�
				\onslide<8->{
					\fill [green] (6.3,-4.4) rectangle (7.3,-4.6);
					\draw [green] [-|] (7.3,-4.5) -- (9.3,-4.5);
				}
				
				% Lmax
				\onslide<7->{
					\draw [<->] (0,-4.1) -- (0.5,-4.1);
					\node [above] at (0.25,-4.1) {\footnotesize $lmax$};
				}
				\onslide<8->{
					\draw [<->] (6,-4.1) -- (6.5,-4.1);
					\node [above] at (6.25,-4.1) {\footnotesize $lmax$};
				}
				
				% Lmin
				\onslide<7->{
					\draw [<->] (-0.5,-4.9) -- (0.5,-4.9);
					\node [below] at (0,-4.85) {\footnotesize $lmin$};
				}
				\onslide<8->{
					\draw [<->] (6.3,-4.9) -- (7.3,-4.9);
					\node [below] at (6.8,-4.85) {\footnotesize $lmin$};
				}
				
				% Fl�che
				\onslide<9->{
					\draw [-triangle 45] (3.4,-6) -- (3.4,-7.5);
				}
				
				% Contig obtenu
				\onslide<9->{
					\draw [green] [|-] (-2.5,-8.5) -- (-1,-8.5);
					\fill [green] (-1,-8.4) rectangle (0,-8.6);
					\draw [green] [-] (0,-8.5) -- (1,-8.5);
					\fill [green] (1,-8.4) rectangle (2,-8.6);
					\fill[fill=black] (2,-8.4) rectangle (5,-8.6);
					\draw [green] [-] (5,-8.5) -- (6.5,-8.5);
					\fill [green] (6.5,-8.4) rectangle (7.3,-8.6);
					\draw [green] [-|] (7.3,-8.5) -- (9.3,-8.5);
					\node [right] at (9.3,-8.5) {\footnotesize obtained contig};
				}
				
				% Accolade
				\onslide<10->{
					\draw [decorate,decoration={brace, amplitude=10pt}]
					(14,-3.4) -- (14,-8.5) node [black, midway, xshift=2cm] 
					{\footnotesize Step 2};
				}
			\end{tikzpicture}
		}
	\end{frame}
	
	\begin{frame}[c]{Results and conclusion}		
		Results of our method on a long reads dataset from ADP1, with parameters set to $lmin = 100$ and $lmax = 10$ : \sl
		
		\onslide<2->{
		\resizebox{\textwidth}{!}{
			\begin{tabular}{|c|c|c|c|c|c|c|c|}
				\hline
				Reads & Average length & Average identity & Number of contigs / read & Average length & Average identity & Covered template & Time \\
				\hline
				1D & 2 052 & 56,5\% & 2,296 & 645 & 88,636\% & 72,17\% & 19min52 \\
				\hline
				2D & 10 033 & 74,5\% & 2,732 & 2 421 & 88,186\% & 65,93\% & 14h06min \\
				\hline
			\end{tabular}
		} \sl
		}
		
		\begin{itemize}
			\onslide<3->{
			\item Runtime : Less than 10 seconds per long read on average \sl
			}
			\onslide<4->{
			\item Still a high error rate of 12 \% \sl
			}
			\onslide<5->{
			\item Production of multiple contigs for each long read, and weak coverage \sl
			}
			\onslide<6->{
			\item Tuning the parameters didn't allow better results
			}
		\end{itemize}		
	\end{frame}
	
	\section{Scaffolding with long reads}
	
	\begin{frame}[c]{Idea}
		\begin{itemize}
			\item Obtain contigs from a short reads assembly \sl
			
			\onslide<2->{
			\item Align long reads on these contigs with BLASR to find local alignments \sl
			}			
			
			\onslide<3->{
			\item Link an order contigs with the help of long reads aligned on multiple contigs \sl
			}
						
			\onslide<4->{
			\item Fill the gaps between two linked contigs with a consensus of the bases from the linking long reads
			}
		\end{itemize}
	\end{frame}
	
	\begin{frame}[c]{Example}
		\resizebox{\textwidth}{!}{
		\begin{tikzpicture}
			\draw [|-|] (-10,1) -- (-3,1);
			\node at (-11,1) {contig$_1$};
			\draw [|-|] (2,1) -- (7,1);
			\node at (8,1) {contig$_2$};
			\onslide<2->{
			\draw [|-|, red] (-5,0) -- (5,0);
			\node at (6,0) {\textcolor{red}{long read$_1$}};
			}
			\onslide<3->{
			\draw [|-|, red] (-7,-1) -- (4,-1);
			\node at (5,-1) {\textcolor{red}{long read$_2$}};
			}
			\onslide<4->{
			\node at (-0.5,-2) {\textcolor{red}{\vdots}};
			}
			\onslide<5->{			
			\draw [|-|, red] (-4,-3) -- (3,-3);
			\node at (4,-3) {\textcolor{red}{long read$_n$}};
			}
			\onslide<6->{
			\draw [dashed, red] (-3,1) -- (-3,-3);
			\draw [dashed, red] (2,1) -- (2,-3);
			}
			\onslide<7->{
			\draw [decorate,decoration={brace, amplitude=10pt, mirror}]
					(-3,-3.5) -- (2,-3.5) node [black, midway, yshift=-1cm] 
					{\footnotesize Consensus};
			}
			
			\onslide<8->{			
			\draw [-triangle 45] (-0.5,-5) -- (-0.5,-6);										
			\draw [|-|] (-10,-7) -- (-3,-7);
			\draw [-, red] (-3,-7) -- (2,-7);
			\draw [|-|] (2,-7) -- (7,-7);
			\node at (8.5,-7) {obtained scaffold};
			}
		\end{tikzpicture}
		}
	\end{frame}
	
	\begin{frame}[c]{Observations and conclusion}
		\begin{itemize}
			\onslide<1->{
			\item Already existing method: SSPACE-LongRead already allows scaffolding with long reads, but fills the gaps
				  between the contigs with Ns instead of building a consensus of the long reads bases \sl
			}
							  
			\onslide<2->{
			\item PBJelly also allows scaffolding with long reads, and fills the gaps with bases from the long reads \sl
			}
			
			\onslide<3->{			
			\item Useless to develop this idea any more
			}
		\end{itemize}
	\end{frame}
	
	\section{Comparing k-mers from short reads and from long reads}
	
	\begin{frame}[c]{Idea}
		\begin{itemize}
			\item Get $k$-mers appearing in the long reads, but not in the short reads contigs \sl
			
			\onslide<2->{
			\item Assemble the obtained $k$-mers $\Rightarrow$ Obtain new contigs
			}
			
			\onslide<3->{
			\item Use these new contigs to cover regions from the reference genome that were previously uncovered by the short reads contigs \sl
			}
			
			\onslide<4->{
			\item Try to allow a less fragmented assembly by assembling together short reads and long reads contigs
			}
		\end{itemize}
	\end{frame}
	
	\begin{frame}[c]{Tests and results}
		With a set of reads from ADP1 : \sl
		
		\resizebox{\textwidth}{!}{
		\begin{tabular}{|c|c|c|}
			\hline
			$k$-mer size & Number of $k$-mers & Number of $k$-mer in the short reads contigs \\
			\hline
			64 & 375 999 371 & 380 967 \\
			\hline
			32 & 335 836 269 & 2 942 135 \\
			\hline
			16 & 322 493 916 & 6 197 488 \\
			\hline
			8 & 65 376 & 65 309 \\
			\hline
		\end{tabular}
		} \sl \sl
		
		\onslide<2->{
		Only 8-mers seem to be present enough in the short reads contigs, but too short to bring any interesting information \sl
		}		
		
		\onslide<3->{
		$\Rightarrow$ More interesting to search for spaced $k$-mers ?
		}
	\end{frame}
	
	\begin{frame}[c]{spaced $k$-mers}
		\begin{block}{Definition}
			Here, we define a spaced $k$-mer as a $k$-mer in which we allow a gap of a certain length  
		\end{block} \sl
		
		\onslide<2->{
		\begin{exampleblock}{Example}
			For instance, the 8-mer GATCTTAC, if we allow a gap of length 2, becomes the following spaced "8"-mer : GATC**TTAC,
			where * denote joker positions (match or mismatch allowed)
		\end{exampleblock} \sl
		}
		
		\onslide<3->{
		We use this definition of spaced $k$-mers rather than the classical one in order to take into account indel. error instead
		of substitutions errors, as they are more common in long reads
		}
	\end{frame}
	
	\begin{frame}[c]{Tests and results}
		Results on the previous dataset, allowing a gap of length at most 10 : \sl
		
		\resizebox{\textwidth}{!}{
		\begin{tabular}{|c|c|c|}
			\hline
			$k$-mer size & Number of $k$-mers & Number of $k$-mer in the short reads contigs \\
			\hline
			64 & 375 999 371 & 425 155 \\
			\hline
			32 & 335 836 269 & 3 859 742 \\
			\hline
			16 & 322 493 916 & 16 036 610 \\
			\hline
		\end{tabular}
		} \sl \sl
		
		\onslide<2->{
		$\Rightarrow$ Comparison is still not interesting, even with spaced $k$-mers
		}
	\end{frame}
	
	\begin{frame}[c]{Observations}
		\begin{itemize}
			\onslide<1->{
			\item Trimming the ends of the longs reads doesn't allow any better results
				  $\Rightarrow$ Unlike short reads, errors aren't mostly located at the ends of the reads, but occur
				  everywhere \sl
			}
			
			\onslide<2->{
			\item Comparing long reads $k$-mer directly with short reads $k$-mers doesn't allow
				  any better results \sl
			}
			
			\onslide<3->{
			\item Not interesting approach \sl
			}
			
			\onslide<4->{
			\item Allow more gaps in the spaced $k$-mers, and not just one in the middle, could potentially yield better
				  resultats $\Rightarrow$ Need to find an algorithm that builds a spaced suffix arrays
			}
		\end{itemize}
	\end{frame}
	
	\section{Working correction approach}
	
	\begin{frame}[c]{Principle}	
		5 �tapes : \sl
			
		\begin{enumerate}
			\onslide<2->{
			\item Short reads correction (with Quorum) \sl	
			}
			
			\onslide<3->{
			\item Mapping of the short reads on the long reads, in order to find seeds (with BLASR) \sl
			}
			
			\onslide<4->{
			\item Merging of the seeds that overlap over more than a certain length \sl
			}
			
			\onslide<5->{
			\item Linking of the seeds, by extending them with perfect overlaps with $k$-mers from the corrected short reads \sl
			}
			
			\onslide<6->{
			\item Extension of the obtained synthetic long read, on the left (resp. right) of the leftmost (resp. rightmost) seed
			}
		\end{enumerate}				
	\end{frame}
	
	\begin{frame}[c]{Used tool : PgSA}
		\onslide<1->{
		PgSA (Pseudogenome Suffix Array) allows the indexing of a set of reads, in order to answer the 7 following queries, for a given string $f$ :
		}
		
		\begin{enumerate}
			\onslide<2->{
			\item In which reads does $f$ occur?
			}
			\onslide<3->{
			\item In how many reads does $f$ occur?
			}
			\onslide<4->{
			\item What are the occurrence positions of $f$?
			}
			\onslide<5->{
			\item What is the number of occurrences of $f$?
			}
			\onslide<6->{
			\item In which reads does $f$ occur only once?
			}
			\onslide<7->{
			\item In how many reads does $f$ occur only once?
			}
			\onslide<8->{
			\item What are the occurrence positions of $f$ in the reads where it occurs only once?
			}
		\end{enumerate}
		
		\onslide<9->{
		Among these requests, the third one will allow us to find perfect overlaps between $k$-mers
		}
	\end{frame}
	
	\begin{frame}[c]{Used tool : PgSA}
		Other data structures (Gk-Arrays, Compressed Gk-Arrays) are able to handle these requests, but the length $k$ of the string $f$
		has to bet set at compilation time, whereas PgSA allows the handling of the requests for arbitrary values of $k$ \sl		
		
		\onslide<2->{
		$\Rightarrow$ Allow to look for overlaps of length $k-2$ if no overlap of length $k-1$ was found, without any need to recompute the whole index
		}
	\end{frame}		
	
	\begin{frame}[c]{Step 4}
		
		The $k$-mers from the set of correct short reads (and their reverse-complements) are indexed with PgSA, and third request is looped over
		to find perfect overlap between $k$-mers, allowing to link together the seeds \sl		
		
			\onslide<2->{
			\begin{center}
			\resizebox{0.75\textwidth}{!}{
				\begin{tikzpicture}
					%TEMPLATE
					\draw [fill,blue] (-6,1.05) rectangle (5.5,0.95);
					\node at (6,1) {\textbf{\textcolor{blue}{.  .  .  .}}};
					\draw [fill=blue] [-|] (6.5,1.05) rectangle (7.5,0.95);
					\node [right] at (8,1) {\textcolor{blue}{long read}};
					%SEEDS					
					\onslide<1-9>{
					\draw [fill=black] (-5.5,0.05) rectangle (-2.5,-0.05);
					}					
					\onslide<3-5>{
					\draw [fill=red] (-3.5,0.05) rectangle (-2.5,-0.05);
					}
					\onslide<5-8>{
					\draw [fill=green] (-2.5,0.05) rectangle (-2.25,-0.05);
					}
					\onslide<6-8>{
					\draw [fill=green] (-3.25,0.05) rectangle (-2.25,-0.05);
					}
					\onslide<8>{
					\draw [fill=orange] (-2.25,0.05) rectangle (-2,-0.05);
					}
					\onslide<9>{
					\draw [fill=orange] (-3,0.05) rectangle (-2,-0.05);
					}
					\draw [fill=black] (2,0.05) rectangle (5,-0.05);
					%K-MERS
					\onslide<4>{
					\draw [fill=green] (-3.25,-0.1) rectangle (-2.25,-0.2);
					}
					\onslide<7>{
					\draw [fill=orange] (-3,-0.1) rectangle (-2,-0.2);
					}
					\onslide<10-12>{
					\draw [fill=black] (-5.5,0.1) rectangle (2.75,0.2);
					}
					\onslide<11-12>{
					\draw [fill=yellow] (1.75,0.1) rectangle (2.75,0.2);
					}
					\onslide<12>{
					\draw [fill=pink] (2,0.05) rectangle (3,-0.05);
					}
					\onslide<13->{
					\draw [fill=black] (-5.5,0.05) rectangle (3,-0.05);
					}
				\end{tikzpicture}
			}
			\end{center} \sl
			}
	\end{frame}
	
	\begin{frame}[c]{Observations}
		\begin{itemize}				  
			\item It possible that a given $k$-mer perfectly overlaps multiple other $k$-mers $\Rightarrow$ Exploration of every possible
				  extension with backtracking \sl
			
			\onslide<2-> {
			\item Some seeds might be impossible to link together $\Rightarrow$ Production of a synthetic long read fragmented
				  in multiple parts \sl
			}
			
			\begin{center}
			\resizebox{0.75\textwidth}{!}{
				\begin{tikzpicture}
					%TEMPLATE
					\onslide<3->{
					\draw [fill=blue] [-] (-12,1.05) rectangle (9,0.95);
					\node [right] at (9,1) {\textcolor{blue}{long read}};
					}
					%SEEDS
					\onslide<4->{
					\draw [fill=black] (-9,0.05) rectangle (-6,-0.05);
					\node [right] at (-6,0) {seed$_1$};
					}
					\onslide<5->{
					\draw [fill=black] (3,0.05) rectangle (6,-0.05);
					\node [right] at (6,0) {seed$_2$};
					}
					
					%TEMPLATE
					\onslide<6->{
					\draw [fill=red] (-12,-2.05) rectangle (-10.5,-1.95);
					\node at (-10,-2) {\textbf{\textcolor{red}{. . . .}}};
					\draw [fill=red] (-9.5,-2.05) rectangle (6,-1.95);
					\node at (6.5,-2) {\textbf{\textcolor{red}{. . . .}}};
					\draw [fill=red] (7,-2.05) rectangle (9,-1.95);
					\node [right] at (9,-2) {\textcolor{red}{reference genome}};
					}
					%SEEDS
					\onslide<7->{					
					\draw [fill=black] (-8,-3.05) rectangle (-5,-2.95);
					\node [right] at (-5,-3) {seed$_2$};
					}
					\onslide<8->{
					\draw [fill=black] (2.5,-3.05) rectangle (5.5,-2.95);
					\node [right] at (5.5,-3) {seed$_1$};
					}	
				\end{tikzpicture}
			}
			\end{center}
			
			\onslide<9->{
			\item When only one seed mapped on a given long read, we just extend this seed on both directions
			}
		\end{itemize}
	\end{frame}
	
	\begin{frame}[c]{Step 5}
		The leftmost seed doesn't always map right at the beginning of the long read, and the rightmost doesn't 
		always reach the end \sl
		
		\onslide<2->{
		$\Rightarrow$ Once all seeds have been linked, and the synthetic long read produced, we extend, with the help
		of perfectly overlapping $k$-mers, its both ends, until the borders of the initial long reads are reached,
		or an ambiguity is met (multiple $k$-mers perfectly overlapping the currently considered one)
		}
	\end{frame}
	
	\begin{frame}[c]{Step 5, first case : No ambiguity}
			\begin{center}
			\resizebox{0.75\textwidth}{!}{
				\begin{tikzpicture}
					%TEMPLATE
					\draw [fill=blue] [-] (-12,1.05) rectangle (9,0.95);
					\node [right] at (9,1) {\textcolor{blue}{long read}};
					%LR CORRIG�
					\draw [fill=black] [-] (-9,-0.95) rectangle (6,-1.05);
					\onslide<2-4>{
					\draw [fill=red] [-] (5,-0.95) rectangle (6,-1.05);
					}
					\node [right] at (0,-2) {synthetic long read};
					%EX
					\onslide<3>{
					\draw [fill=green] (5.25,-1.15) rectangle (6.25,-1.25);
					}
					\onslide<4-5>{
					\draw [fill=green] (6,-0.95) rectangle (6.25,-1.05);
					}
					\onslide<5-7>{
					\draw [fill=green] (5.25,-0.95) rectangle (6.25,-1.05);
					}
					\onslide<6>{
					\draw [fill=orange] (5.5,-1.15) rectangle (6.50,-1.25);
					}
					\onslide<7-8>{
					\draw [fill=orange] (6.25,-0.95) rectangle (6.50,-1.05);
					}
					\onslide<8>{
					\draw [fill=orange] (5.50,-0.95) rectangle (6.50,-1.05);
					}
					\onslide<9->{
					\draw [fill=black] [-] (-9,-0.95) rectangle (8.75,-1.05);
					}
					\onslide<10-12>{
					\draw [fill=yellow] [-] (7.75,-0.95) rectangle (8.75,-1.05);
					}
					\onslide<11>{
					\draw [fill=pink] (8,-1.15) rectangle (9,-1.25);
					}
					\onslide<12-13>{
					\draw [fill=pink] (8.75,-0.95) rectangle (9,-1.05);
					}
					\onslide<13->{
					\draw [fill=pink] (8,-0.95) rectangle (9,-1.05);
					}				
					%TRAIT
					\onslide<14->{
					\draw [green, dashed] (9,1.05) -- (9,-1.25);
					}
					\onslide<15->{
					\draw [fill=black] [-] (-9,-0.95) rectangle (9,-1.05);
					}
				\end{tikzpicture}
			} 
			\end{center}
	\end{frame}
	
	\begin{frame}[c]{Step 5, second case : Ambiguity}
			\begin{center}
			\resizebox{0.75\textwidth}{!}{
				\begin{tikzpicture}
					%TEMPLATE
					\draw [fill=blue] [-] (-12,1.05) rectangle (9,0.95);
					\node [right] at (9,1) {\textcolor{blue}{long read}};
					%LR CORRIG�
					\draw [fill=black] [-] (-9,-0.95) rectangle (6,-1.05);
					\onslide<2-4>{
					\draw [fill=red] [-] (-9,-0.95) rectangle (-8,-1.05);
					}
					\node [right] at (0,-2) {synthetic long read};
					%EXT GAUCHE
					\onslide<3>{
					\draw [fill=green] (-9.25,-1.15) rectangle (-8.25,-1.25);
					}
					\onslide<4-5>{
					\draw [fill=green] (-9.25,-0.95) rectangle (-9,-1.05);
					}
					\onslide<5-7>{
					\draw [fill=green] (-9.25,-0.95) rectangle (-8.25,-1.05);
					}
					\onslide<6>{
					\draw [fill=orange] (-9.50,-1.15) rectangle (-8.50,-1.25);
					}
					\onslide<7-8>{
					\draw [fill=orange] (-9.50,-0.95) rectangle (-9.25,-1.05);
					}
					\onslide<8-9>{
					\draw [fill=orange] (-9.50,-0.95) rectangle (-8.50,-1.05);
					}
					\onslide<9>{
					\draw [fill=blue] (-9.75,-1.15) rectangle (-8.75,-1.25);
					\draw [fill=pink] (-9.75,-0.85) rectangle (-8.75,-0.75);
					}
					\onslide<10>{
					\draw [fill=black] [-] (-9.5,-0.95) rectangle (6,-1.05);
					}
				\end{tikzpicture}
			} 
			\end{center}
	\end{frame}
	
	\begin{frame}[c]{Results and comparison with NaS : Mapping}		
	
			On the 6 ADP1 long reads datasets available from the Genoscope website:
			
			\begin{table}
			\resizebox{1\textwidth}{!}{
			\begin{tabular}{|c|c|c|c|c|c|}
				\hline
				& \textbf{Number of reads} & \textbf{Average length} & \textbf{Cumulative size} & \textbf{Average identity} &s \textbf{Runtime} \\
				\hline
				\textbf{Raw LRs} & 70 314 & 	2 530 & 177 869 033 & 3,84 \% & N.A. \\
				\hline
				\textbf{NaS (fast)} & 8 219 & 4 514 & 37 099 564 & 99,92 \% & TODO \\
				\hline
				\textbf{NaS (sensitive)} & 12 053 & 6 338 & 76 388 104 & 99,89 \% & TODO \\
				\hline
				\textbf{Nous} & 7 425 (249 fragmented) & 10 250 & 78 739 767 & 99,57 \% & TODO \\
				\hline
			\end{tabular}
			}
			\caption{On the 1D long reads}
			\end{table}
		
			\onslide<2->{
			\begin{table}
			\resizebox{1\textwidth}{!}{
			\begin{tabular}{|c|c|c|c|c|c|}
				\hline
				& \textbf{Number of reads} & \textbf{Average length} & \textbf{Cumulative size} & \textbf{Average identity} &s \textbf{Runtime} \\
				\hline
				\textbf{Raw LRs} & 18 697 & 10 884 & 203 496 742 & 37,07 \% & N.A. \\
				\hline
				\textbf{NaS (fast)} & 15 844 & 11 084 & 175 607 625 & 99,78 \% & TODO \\
				\hline
				\textbf{NaS (sensitive)} & 16 439 & 11 871 & 195 138 674 & 99,79 \% & TODO \\
				\hline
				\textbf{Nous} & 15 575 (984 fragmented) & 10 562 & 178 222 404 & 99,54 \% & TODO \\
				\hline
			\end{tabular}
			}
			\caption{On the 2D long reads}
			\end{table}
			}
	\end{frame}
	
	\begin{frame}[c]{Results and comparison with NaS : Assembly}		
			With the two 1D and 2D datasets previously corrected:			
			
			\begin{table}
			\resizebox{\textwidth}{!}{
			\begin{tabular}{|c|c|c|c|c|c|}
			\hline
			\textbf{Tool} & \textbf{Number of reads} & \textbf{Number of contigs} & \textbf{Genome coverage} 	
			& \textbf{Identity} \\
			\hline
			\textbf{NaS (fast)} & 24,063 & 1 & 100 \% & 99.98 \% \\
			\textbf{NaS (sensitive)} & 28,492 & 1 & 100 \% & 99.99 \% \\
			\textbf{Our method} & TODO & TODO & TODO & TODO \\
			\hline
			\end{tabular}
			}
			\caption{Assembly results \label{tabres3}}
			\end{table}
			
	\end{frame}
	
	\begin{frame}[c]{To do}
		\begin{itemize}
			\item Try to produce less fragmented synthetic long reads $\Rightarrow$ Tuning of BLASR parameters \sl
			
			\onslide<2->{
			\item Allow the correction to run in parallel to reduce runtime
			}
		\end{itemize}
	\end{frame}
	
	\section{Long reads mapping on a DBG}
	
	\begin{frame}[c]{M�thodes existantes}
		\begin{itemize}
			\item LoRDEC (2014, Hybrid, short reads DBG) \sl
			
			\item Jabba (2016, Hybrid, short reads DBG) \sl
			
			\item LoRMA (2016, Only LRs, long reads DBG) \sl
			\end{itemize}
	\end{frame}
	
\end{document}