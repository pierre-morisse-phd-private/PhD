\documentclass{article}

\usepackage{ALgo}

\def\iit#1{\mbox{\it #1\/}}
\def\card{\iit{card}}
\def\ew{\varepsilon}
\def\sa#1{\mbox{\tt #1}}

\def\iter{{\mathit iter}}
\def\Prev{{\mathit Prev}}
\def\rank{{\mathit rank}}
\def\Elt{{\mathit Elt}}
\def\groupj{{\mathit groupj}}
\def\length{{\mathit length}}
\def\lastgroup{{\mathit lastgroup}}
\def\firstgroup{{\mathit firstgroup}}
\def\nextgroup{{\mathit nextgroup}}
\def\newgroup{{\mathit newgroup}}
\def\SA{{\mathit SA}}

\begin{document}

%----------------------------------------------------------------------
% TITLE
%----------------------------------------------------------------------
\title{Linear computation of the Lyndon array and the suffix array
}

\author{
Richard Groult
\and
Nicolas Guiomard-Kagan
\and
Thierry Lecroq
\and
Arnaud Lefebvre
\and
Pierre Morisse
\and
\'Elise Prieur-Gaston
\and
Tatiana Rocher
\and
Mikaël Salson
}

\date{}

\maketitle

%----------------------------------------------------------------------
% ABSTRACT
%----------------------------------------------------------------------
\begin{abstract}
\end{abstract}

%----------------------------------------------------------------------
% KEYWORDS
%----------------------------------------------------------------------
%\begin{keywords}
%String algorithms.
%\end{keywords}

%----------------------------------------------------------------------
% SECTION
%----------------------------------------------------------------------
\section{Introduction}
\label{sec:intro}

%----------------------------------------------------------------------
% SECTION
%----------------------------------------------------------------------
\section{Background and basic string definitions}
\label{sec:background}

A {\em string} is a sequence of zero or more symbols from an
 alphabet $\Sigma$; the string with zero symbols is denoted by
 $\varepsilon$.
The set of all strings over the alphabet $\Sigma$ is
 denoted by $\Sigma^*$.
We consider an alphabet of size $s$;
 for $1 \leq i \leq s$, $\sigma[i]$ denotes the $i$-th symbol of
 $\Sigma$.
A string $x$ of length $n$ is represented
 by $x[1..n]$, where $x[i]\in\Sigma$ for $1\le i\le n$.
A string $u$ is a {\em prefix} of $x$ if $x=uw$ for $w\in\Sigma^*$.
Similarly, $u$ is a {\em suffix} of $x$ if $x=wu$ for
 $w\in\Sigma^*$.
A string $u$ is a {\em border} of $x$ if $u$ is a prefix and a
 suffix of $x$ and $u \neq x$.

%----------------------------------------------------------------------
% SECTION
%----------------------------------------------------------------------
\section{Baier's algorithm}
\label{sec:baier}

%----------------------------------------------------------------------
% SECTION
%----------------------------------------------------------------------
\section{Our algorithm}
\label{sec:algo}

\begin{algo}{Phase1}{\lastgroup,P}
  \SET{\iter}{1}
  \SET{\Prev[0]}{-1}
  \DOFOR{G=\lastgroup; G \ne \emptyset; G = G.next}
    \SET{\rank}{G.rank}
    \DOFOR{\Elt=G.head; \Elt \ne \emptyset; \Elt = \Elt.next}
      \SET{i}{\Elt.pos}
      \SET{j}{i-1}
      \DOWHILE{j \ne -1 \mbox{ and } P[j].group.rank \le \rank}
        \SET{j}{\Prev[j]}
      \OD
      \SET{\Prev[i]}{j}
      \IF{j = -1}
        \SET{\groupj}{P[j].group}
        \SET{\length}{\groupj.length+G.length}
        \SET{\nextgroup}{\groupj.next}
        \IF{\nextgroup.iter = iter \mbox{ and } \nextgroup.length = length}
          \CALL{Move}{P[j],\groupj,\nextgroup}
        \ELSEIF{\groupj.size = 1}
          \SET{\groupj.length}{\length}
        \ELSE
          \SET{\newgroup}{\CALL{NewGroupBefore}{\nextgroup,\iter,\length}}
          \CALL{MoveDown}{P[j],\groupj,\newgroup}
        \FI
      \FI
    \OD
    \INC{\iter}
  \OD
\end{algo}


\begin{algo}{Phase2}{\firstgroup,P,n}
  \IF{P[n].group.size > 1}
    \SET{\newgroup}{\CALL{NewGroupBefore}{P[n].group}}
    \CALL{MoveUp}{P[n],P[n].group,\newgroup}
    \SET{\SA[\newgroup.rank]}{n}
  \FI
  \DOFOR{G=\firstgroup; G \ne \emptyset; G = G.next}
    \DOFOR{\Elt=G.head; \Elt \ne \emptyset; \Elt = \Elt.next}
      \SET{i}{\Elt.pos}
      \IF{P[i].group.size = 1}
        \SET{\SA[P[i].group.rank]}{i}
      \FI
      \SET{j}{i-1}
      \IF{P[j].group.size > 1}
        \SET{\newgroup}{\CALL{NewGroupBefore}{P[j].group}}
        \CALL{MoveUp}{P[j],P[j].group,\newgroup}
        \SET{\SA[\newgroup.rank]}{j}
      \FI
    \OD
  \OD
  \RETURN{\SA}
\end{algo}

Example

\begin{verbatim}
0 1 2 3 4 5 6 7 8 9 10
m i s s i s s i p p i

iter = 0, rank = 0, size = 4, length = 1, list = (1,4,7,10), context = i
iter = 0, rank = 4, size = 1, length = 1, list = (0), context = m
iter = 0, rank = 5, size = 2, length = 1, list = (8,9), context = p
iter = 0, rank = 7, size = 4, length = 1, list = (2,3,5,6), context = s

Prev[2] = 1

iter = 0, rank = 0, size = 3, length = 1, list = (4,7,10), context = i
iter = 1, rank = 3, size = 1, length = 2, list = (1), context = is
iter = 0, rank = 4, size = 1, length = 1, list = (0), context = m
iter = 0, rank = 5, size = 2, length = 1, list = (8,9), context = p
iter = 0, rank = 7, size = 4, length = 1, list = (2,3,5,6), context = s

Prev[3] = Prev[2] = 1

iter = 0, rank = 0, size = 3, length = 1, list = (4,7,10), context = i
iter = 1, rank = 3, size = 1, length = 3, list = (1), context = iss
iter = 0, rank = 4, size = 1, length = 1, list = (0), context = m
iter = 0, rank = 5, size = 2, length = 1, list = (8,9), context = p
iter = 0, rank = 7, size = 4, length = 1, list = (2,3,5,6), context = s

Prev[5] = 4

iter = 0, rank = 0, size = 2, length = 1, list = (7,10), context = i
iter = 1, rank = 2, size = 1, length = 2, list = (4), context = is
iter = 1, rank = 3, size = 1, length = 3, list = (1), context = iss
iter = 0, rank = 4, size = 1, length = 1, list = (0), context = m
iter = 0, rank = 5, size = 2, length = 1, list = (8,9), context = p
iter = 0, rank = 7, size = 4, length = 1, list = (2,3,5,6), context = s

Prev[6] = Prev[5] = 4

iter = 0, rank = 0, size = 2, length = 1, list = (7,10), context = i
iter = 1, rank = 3, size = 2, length = 3, list = (1,4), context = iss
iter = 0, rank = 4, size = 1, length = 1, list = (0), context = m
iter = 0, rank = 5, size = 2, length = 1, list = (8,9), context = p
iter = 0, rank = 7, size = 4, length = 1, list = (2,3,5,6), context = s

Prev[8] = 7

iter = 0, rank = 0, size = 1, length = 1, list = (10), context = i
iter = 2, rank = 1, size = 1, length = 2, list = (7), context = ip
iter = 1, rank = 3, size = 2, length = 3, list = (1,4), context = iss
iter = 0, rank = 4, size = 1, length = 1, list = (0), context = m
iter = 0, rank = 5, size = 2, length = 1, list = (8,9), context = p
iter = 0, rank = 7, size = 4, length = 1, list = (2,3,5,6), context = s

Prev[9] = Prev[8] = 7

iter = 0, rank = 0, size = 1, length = 1, list = (10), context = i
iter = 2, rank = 1, size = 1, length = 3, list = (7), context = ipp
iter = 1, rank = 3, size = 2, length = 3, list = (1,4), context = iss
iter = 0, rank = 4, size = 1, length = 1, list = (0), context = m
iter = 0, rank = 5, size = 2, length = 1, list = (8,9), context = p
iter = 0, rank = 7, size = 4, length = 1, list = (2,3,5,6), context = s

Prev[0] = -1

Prev[1] = Prev[0] = -1

Prev[4] = Prev[3] = Prev[1] = -1

Prev[7] = Prev[6] = Prev[4] = -1

Prev[10] = Prev[9] = Prev[7] = -1
\end{verbatim}

%----------------------------------------------------------------------
% SECTION
%----------------------------------------------------------------------
\section{Correctness}
\label{sec:correc}

%----------------------------------------------------------------------
% SECTION
%----------------------------------------------------------------------
\section{Experiments}
\label{sec:expe}

%----------------------------------------------------------------------
% SECTION
%----------------------------------------------------------------------
\section{Conclusion}
\label{sec:conclu}

\end{document}
